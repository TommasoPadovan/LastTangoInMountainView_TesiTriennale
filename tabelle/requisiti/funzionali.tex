\newcolumntype{s}{>{\hsize=.37\hsize}X}
\newcolumntype{f}{>{\hsize=.42\hsize}X}
\newcolumntype{m}{>{\hsize=.21\hsize}X}

\begin{longtable}{s X m}  
			%\rowcolor{orange!85}Codice & Nome & Fonte & Descrizione \\
\endhead
\hline\hline
	\textbf{Requisito} & \textbf{Descrizione} & \textbf{Fonti}\\
\hline
	RFO-1 &
	L'utente può effettuare una nuova rilevazione &
	UC1 \\
\hline
	RFO-1.1 &
	L'utente può riprendere, mediante la pressione di un tasto, il \emph{Point Cloud} attualmente inquadrato. &
	UC1.1 \\
\hline
	RFO-1.2 &
	Il \emph{Point Cloud} catturato deve essere correttamente aggiunto alla ricostruzione corrente. &
	UC 1.1 \\
\hline
	RFO-1.3 &
	L'utente può modificare la visualizzazione del \emph{Point Cloud} a piacimento e scegliere tra la visualizzazione in tempo reale del sensore di profondità e quella della ricostruzione salvata. &
	UC1.2, UC1.3 \\
\hline
	RFO-1.4 &
	Il sistema è in grado di scartare la ricostruzione corrente ed iniziarne una nuova &
	UC1.4\\
\hline
	RFD-1.5 &
	Dopo aver scartato una ricostruzione il sistema è in grado di iniziare la successiva senza dover ripetere le operazioni di localizzazione. &
	UC1.4\\
\hline
	RFO-1.6 &
	L'utente può inviare i dati al \emph{Server}. &
	UC1.5, UC2.3\\
\hline
	RFD-1.7 &
	Nel caso in cui non sia disponibile la connessione internet mentre l'utente sta cercando di inviare la ricostruzione corrente al \emph{Server} deve essere mostrato un opportuno messaggio d'errore. &
	UC1.5, UC5\\
\hline
	RFO-1.8 &
	Il sistema può salvare i dati della ricostruzione corrente su disco nella cartella interna dell'applicazione, il formato deve essere \texttt{pcd}. &
	UC1.6 \\
\hline
	RFO-1.9 &
	Il sistema deve calcolare in tempo reale le principali statistiche riguardanti: posizione del dispositivo, ricostruzione corrente e nuvola di punti inquadrata. &
	UC1.7\\
\hline
	RFO-1.10 &
	Il sistema deve permettere operazioni di undo &
	UC1.8\\
\hline
	RFO-2 &
	Il sistema deve permettere operazioni sui \emph{file} \texttt{pcd} salvati su disco. &
	UC2\\
\hline
	RFO-2.1 &
	Il sistema deve essere in grado di fornire la lista di tutti i \emph{Point Cloud}  salvati. &
	UC2.1\\
\hline
	RFO-2.2 &
	Il sistema deve essere in grado di aprire un \emph{file} \texttt{pcd} e caricarlo come ricostruzione corrente &
	UC2.2\\	
\hline
	RFO-2.3 &
	Il sistema deve essere in grado di eliminare un \emph{Point Cloud} salvato. &
	UC2.\\
\hline
	RFO-3 &
	Il sistema deve essere in grado di permettere operazioni sui file di \emph{mesh} salvati su disco. &
	UC3\\	
\hline
	RFO-3.1 &
	Il sistema deve essere in grado di fornire la lista di tutti le \emph{mesh}  salvati. &
	UC3.1\\
\hline
	RFO-3.2 &
	Il sistema deve essere in grado di dare la possibilità di scaricare le \emph{mesh} elaborate dal \emph{Server}. &
	UC3.2\\
\hline
	RFD-3.3 &
	Nel caso in cui non sia disponibile la connessione internet mentre l'utente sta cercando di scaricare la lista di \emph{mesh} dal \emph{Server} deve essere mostrato un opportuno messaggio d'errore. &
	UC3.2, UC5\\
\hline
	RFO-4 &
	L'applicazione deve fornire una interfaccia che permetta all'utente di svolgere semplicemente tutte le operazioni riportate nei casi d'uso. &
	UC0, interna \\
\hline
	RFO-4.1 &
	L'interfaccia deve fornire un insieme di pulsanti per permette all'utente di impartire ordini al sistema. &
	UC1.2 \newline UC1.3 \newline UC1.5 \newline UC1.6 \newline UC1.8 \newline UC1.11 \newline UC1.12 \newline UC2.2 \newline UC2.3 \newline UC2.4 \newline UC2.5 \newline UC3.2 \newline UC3.3 \newline UC3.4 \newline UC3.5  \\
\hline
	RFO-4.1.1 &
	L'interfaccia deve fornire un pulsante per permettere la registrazione di un singolo \emph{Point Cloud}. &
	UC1.1 \\
\hline
	RFO-4.1.2 &
	L'interfaccia deve fornire un pulsante per permettere di passare con il \emph{render} dalla visione in prima alla visione in terza persona e viceversa.&
	UC1.2 \\
\hline
	RFO-4.1.3 &
	L'interfaccia deve fornire un interruttore per permettere di alternare tra la visualizzazione in tempo reale e quella dell'oggetto ricostruito. &
	UC1.3 \\
\hline
	RFO-4.1.4 &
	L'interfaccia deve fornire un pulsante per permettere il reset della ricostruzione. &
	UC1.4 \\
\hline
	RFO-4.1.5 &
	L'interfaccia deve fornire un pulsante per permettere l'invio dei dati al \emph{Server}. &
	UC1.5 \\
\hline
	RFO-4.1.6 &
	L'interfaccia deve fornire un pulsante per permettere il salvataggio dei dati su disco. &
	UC1.6 \\
\hline
	RFO-4.1.7 &
	L'interfaccia deve fornire un pulsante per permettere le operazioni di undo. &
	UC1.8 \\
\hline
	RFO-4.1.8 &
	L'interfaccia deve fornire un pulsante per permettere di passare alla visualizzazione dei file contenenti i e\emph{Point Cloud}. &
	UC1.11 \\
\hline
	RFO-4.1.9 &
	L'interfaccia deve fornire un pulsante per permettere di passare alla visualizzazione dei file contenenti le \emph{mesh}. &
	UC1.12 \\
\hline
	RFO-4.1.10 &
	L'interfaccia deve fornire un pulsante per permettere il caricamento di un \emph{Point Cloud} come ricostruzione attuale. &
	UC2.2 \\
\hline
	RFO-4.1.11 &
	L'interfaccia deve fornire un pulsante per permettere l'invio al \emph{Server} di un \emph{Point Cloud} dall'\emph{activity} che lista i \emph{file} \texttt{pcd}.&
	UC2.3 \\
\hline
	RFO-4.1.12 &
	L'interfaccia deve fornire un pulsante per permettere l'eliminazione di un \emph{Point Cloud} salvato su disco. &
	UC2.4 \\
\hline
	RFO-4.1.13 &
	L'interfaccia deve fornire un pulsante per permettere il ritorno dalla lista dei \emph{Point Cloud} all'\emph{activity} principale. &
	UC2.5 \\
\hline
	RFO-4.1.14 &
	L'interfaccia deve fornire un pulsante per permettere di scaricare delle \emph{mesh} elaborate dal \emph{Server}. &
	UC3.2 \\
\hline
	RFO-4.1.15 &
	L'interfaccia deve fornire un pulsante per permettere di eliminare una \emph{mesh} salvata su disco. &
	UC3.4 \\
\hline
	RFO-4.1.16 &
	L'interfaccia deve fornire un pulsante per permettere il ritorno dalla lista delle \emph{mesh} all'\emph{activity} &
	UC3.5 \\
\hline
	RFO-4.2 &
	L'interfaccia deve fornire delle statistiche riguardanti il \emph{Point Cloud} in tempo reale e la ricostruzione corrente. &
	UC1.7\\
\hline
	RFO-4.3 &
	L'interfaccia deve fornire opportuni strumenti per visualizzare dati dei sensori e le varie ricostruzioni in maniera grafica. &
	UC1.9 \newline UC1.10 \newline UC3.3\\
\hline
	RFO-4.3.1 &
	L'interfaccia deve fornire la possibilità di visualizzare sullo schermo del dispositivo la \emph{preview} della fotocamera a colori. &
	UC1.10\\
\hline
	RFO-4.3.2 &
	L'interfaccia deve fornire la possibilità di visualizzare sullo schermo del dispositivo un \emph{render} di tipo \emph{OpenGL} in grado di mostrare \emph{Point Cloud}. &
	UC1.9\\
\hline
	RFO-4.3.2.1 &
	Il render deve permettere l'operazione di rotazione quando possibile tramite \emph{swipe} del dito. &
	UC1.9\\
\hline
	RFO-4.3.2.2 &
	Il render deve permettere l'operazione di zoom tramite \emph{pinch} delle dita. &
	UC1.9\\	
\hline
	RFO-4.3.3 &
	L'interfaccia deve fornire la possibilità di visualizzare sullo schermo del dispositivo un \emph{render} per le \emph{mesh} 3D. &
	UC3.3\\
\hline
	RFO-4.3.3.1 &
	Il render deve permettere l'operazione di rotazione quando possibile tramite \emph{swipe} del dito. &
	UC3.3\\
\hline
	RFO-4.3.3.2 &
	Il render deve permettere l'operazione di zoom tramite \emph{pinch} delle dita. &
	UC3.3\\
\hline
\bottomrule
\caption{Tabella del tracciamento dei requisti funzionali}\label{tab:requisiti-funzionali}
\end{longtable}










