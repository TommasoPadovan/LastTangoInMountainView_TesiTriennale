\newcolumntype{s}{>{\hsize=.37\hsize}X}
\newcolumntype{f}{>{\hsize=.42\hsize}X}
\newcolumntype{m}{>{\hsize=.21\hsize}X}

\begin{longtable}{s X m}  
			%\rowcolor{orange!85}Codice & Nome & Fonte & Descrizione \\
\endhead
\hline\hline
	\textbf{Requisito} & \textbf{Descrizione} & \textbf{Fonti}\\
\hline
	RPO-1 &
	Le operazioni di ricostruzione e visualizzazione del \emph{Point Cloud} devono essere svolte efficientemente &
	Richiesta committente \\
\hline
	RPO-1.1 &
	L'elaborazione del \emph{Point Cloud} deve essere svolta in un tempo finito e senza grosse variazioni tra una rilevazione e l'altra. &
	UC1.1 \\
\hline
	RPD-1.2 &
	L'elaborazione del \emph{Point Cloud} deve essere sufficientemente ottimizzata da poter permettere almeno 5-6 catture al secondo. &
	UC1.1 \\
\hline
	RPD-1.3 &
	Il cambio di modalità del render deve essere effettuato senza \emph{delay} in quanto è una operazione molto frequente durante la rilevazione. &	
	UC1.2 \newline UC 1.3 \\
\hline


	RPD-2 &
	I \emph{file} e le strutture dati utilizzate per salvare e spedire le ricostruzioni ed i singoli \emph{Point Cloud} devono essere ottimizzati. &
	Richiesta committente \newline analisi dei rischi\\
\hline
	RPD-2.1 &
	I \emph{file} \texttt{pcd} generati devono essere di dimensioni ridotte e non devono presentare punti uguali ripetuti. &
	UC1.5 \newline UC1.6 \newline UC2\\
\hline
	RPD-2.2 &
	I pacchetti da inviare al \emph{Server} devono essere di dimensioni adeguate ed il \emph{file} da inviare deve essere diviso e non essere inviato tutto in una volta. &
	UC 1.6 \newline UC2.3 \newline UC3.2\\
\hline
	RPD-3 &
	Il \emph{render} dei \emph{Point Cloud} deve presentarsi fluido, non scattoso e rappresentare sempre quello che il dispositivo sta inquadrando momento per momento. &
	UC1.9\\
\hline
	RPD-4 &
	Il \emph{render} delle \emph{mesh} deve presentarsi fluido e non scattoso. &
	UC3.3\\
\hline
	RPD-5 &
	L'interfaccia deve essere sempre responsiva e non bloccarsi mentre c'è una elaborazione in corso. &
	Richiesta committente\\
\bottomrule
\caption{Tabella del tracciamento dei requisti prestazionali}
\end{longtable}
