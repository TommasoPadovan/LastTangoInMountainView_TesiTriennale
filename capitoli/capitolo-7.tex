% !TEX encoding = UTF-8
% !TEX TS-program = pdflatex
% !TEX root = ../tesi.tex
% !TEX spellcheck = it-IT

%**************************************************************
\chapter{Conclusioni}
\label{cap:conclusioni}
%**************************************************************
Al di là del formalismo informatico, lo scopo principale di questo progetto era indagare sul possibile uso della tecnologia \emph{Tango} nel campo ispettivo come effettivo supporto allo studio di beni materiali. Il prototipo realizzato sembra confermare che cioè è possibile.\\
I risultati ottenuti sono stati piuttosto soddisfacenti e con qualche raffinamento appare possibile inserire l'applicazione in un contesto produttivo.

\section{Prove pratiche}
Il prototipo prodotto è stato testato in numerori ambienti e su diversi oggetti. Nella quasi totalià dei casi i risultati sono stati più che sufficienti per quanto riguarda la qualità del \emph{Point Cloud} ricostruito.\\
Per quanto riguarda invece il calcolo del volume i risultati non sono ancora totalmente sufficienti: il volume ottenuto è sempre dello stesso ordine di grandezza del volume reale, ma spesso è affetto da un errore relativo tra il 30 ed il 50\% ed un errore del genere non è affatto tollerabile. Tale divario però è facilmente appianabile migliorando la qualità delle elaborazioni dei \emph{Point Cloud} e delle \emph{mesh} lato \emph{Server}.

\section{Sviluppi futuri}

\subsection{icp on tablet}
\subsection{texture dei punti}
\subsection{ingrazione cpp lato tablet}
\subsection{rimozione artefatti}

%**************************************************************
\section{Consuntivo finale}

%**************************************************************
\section{Raggiungimento degli obiettivi}

%**************************************************************
\section{Conoscenze acquisite}

%**************************************************************
\section{Valutazione personale}
