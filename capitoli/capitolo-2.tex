% !TEX encoding = UTF-8
% !TEX TS-program = pdflatex
% !TEX root = ../tesi.tex
% !TEX spellcheck = it-IT

%**************************************************************
\chapter{Processi e strumenti}
\label{cap:processi-metodologie}
%**************************************************************

\intro{La buona definizione dei processi produttivi è fondamentale per evitare il rischio di fallimento.}\\

%**************************************************************
\section{Processo sviluppo prodotto}
In ambito aziendale si è scelto di provare, per questo progetto, un processo di sviluppo \emph{software} basato sulla filosofia \emph{Lean}.\\
Dato che l'obiettivo principale era fornire un prototipo ci si è limitati solamente alle tre fasi iniziali dello sviluppo di lean, ovvero \emph{Kick-Off}, \emph{Concept Preview} e \emph{Product Prototype}.


\subsection{Kick-Off}
Questa è la prima fase dello sviluppo \emph{software}, coincide con la prima riunione ufficiale del team di progetto, aperta anche agli \emph{Stakeholder}.\\
Si pone lo scopo di iniziare la fase l'\emph{allestimento} e l'\emph{avviamento} in cui viene determinata la natura e lo scopo del progetto.\\

\subsection{Concept Preview}
Fase in cui è reso disponibile un primo campione di prova del prodotto, detto \emph{concept}. Esso può essere incompleto e affetto da errori, ma deve essere in grado di dimostrare agli \emph{stakeholder} le caratteristiche principali che avrà il prodotto finito.\\
Esso è soggetto a un riesame che ha lo scopo di valutare se è in linea con gli obiettivi definiti nella \emph{Value Proposition}, la \emph{milestone} non può essere raggiunta senza che questo riesame abbia esito positivo.\\
Questa fase coincide anche con l'inizio della progettazione, che deve essere portata avanti fino ad un livello di dettaglio ritenuto opportuno dal \emph{team}.\\

\subsection{Product Prototype}
Fase in cui è messo a disposizione il primo prototipo del nuovo prodotto, completo nelle sue funzioni (sviluppo finito) ma non ancora messo a punto mediante verifiche e correzioni, per garantirne funzionalità e prestazioni. Il prototipo deve essere ad un stato tale da poter essere dato in valutazione agli \emph{stakeholder}.\\
Esso è soggetto a un riesame che ha lo scopo di valutare se è in linea con gli obiettivi definiti sia \emph{Value Proposition} che nella \emph{Requirements Specification}, la \emph{milestone} non può essere raggiunta senza che questo riesame abbia esito positivo.\\
Questa fase coincide con il termine della fase di progettazione e l'inizio della fase di esecuzione, cioè l'insieme dei processi necessari a soddisfare i requisiti del progetto.

\subsection{Fasi successive}
Le fasi successive, ovvero \emph{Product Design Freeze} e \emph{Start Of Production}, possono essere avviate nel futuro a partire dal \emph{Product Prototype} se ciò verrà ritenuto opportuno dall'azienda.



\section{Strumenti}
L'azienda ha lasciato grande libertà riguardo agli strumenti da utilizzare per questo progetto, quindi essi sono stati fissati inizialmente e successivamente incrementati al crescere delle necessità.

\subsection{Codice}
Segue la lista degli strumenti utilizzati per la codifica.
\begin{itemize}
	\item \textbf{Java}: Il linguaggio preferito per l'applicazione lato \emph{tablet}. È stato scelto seguendo le \emph{Best Practice} dello sviluppo \emph{Android}.
	\item \textbf{C++}: Il linguaggio preferito per l'applicazione lato \emph{Server}. È stato scelto perché tutti le più diffuse librerie per l'elaborazione dei \emph{Point Cloud} sono disponibili in questo linguaggio.
	\item \textbf{PCD, Point Cloud Library}: Una delle librerie di maggior rilievo nel campo della \emph{Computer Vision}, mette a disposizione innumerevoli funzionalità per elaborazione ed ottimizzazione dei \emph{Point Cloud}.
	\item \textbf{Tango API}: Le \emph{API} ufficiali per lo sviluppo di applicazioni \emph{Tango}.
	\item \textbf{PHP}: Il linguaggio usato per ricevere ed inviare le richieste \emph{http} quando si necessita comunicazione tra \emph{Server} e dispositivo.
\end{itemize}

\subsection{IDE ed editor}
Segue la lista degli ambienti per la codifica utilizzati durante il progetto.
\begin{itemize}
	\item \textbf{Android Studio}: L'\emph{IDE} ufficiale per le applicazioni \emph{Android}.
	\item \textbf{QT}: L'\emph{IDE} scelto per lo sviluppo del codice \emph{C++}.
	\item \textbf{Sublime Text 2}: L'\emph{editor} di testo usato per scrivere gli \emph{script} \emph{php}.
\end{itemize}

\subsection{Framework}
Segue la lista dei \emph{Framework} usati per durante il tirocinio.
\begin{itemize}
	\item \textbf{Gradle}: È stato usato come \emph{tool} di \emph{build} per tutta l'applicazione \emph{Android}.
	\item \textbf{Rajawali3D 1.0.325}: È stato usato come \emph{framework} grafico per la realizzazione del \emph{render}.
	\item \textbf{OkHttp}: È stata usato come \emph{framework} di riferimento per le richieste \emph{http}, come indicato nella \emph{Android Best Practices}.
	\item \textbf{TangoUx}: \emph{Famework} messo a disposizione dalla \emph{Google} assieme alle \emph{API Tango}. È stato usato per gestire le notifiche all'utente relative ai comportamenti che deve tenere per permettere il buon funzionamento del dispositivo e dei sensori.
	\item \textbf{Jni}: Questo \emph{framework} è stato solamente impostato per sviluppi futuri, permette di implementare metodi \emph{Java} nativamente in \emph{C}. Non è usato nel prototipo presentato.
\end{itemize}












