% !TEX encoding = UTF-8
% !TEX TS-program = pdflatex
% !TEX root = ../tesi.tex
% !TEX spellcheck = it-IT

%**************************************************************
\chapter{Studio di fattibilità ed analisi dei rischi}
\label{cap:descrizione-stage}
%**************************************************************

\intro{Iniziare, quasi da zero, un progetto così impegnativo basandosi su di una tecnologia al limite dell'essere sperimentale espone a gravi rischi di fallimento. Per ciò in questa fase sono state investite molte risorse.}\\

%**************************************************************
\section{Introduzione al progetto}
Data la natura innovativa del progetto è stato necessario produrre diversi prototipi ed effettuare l'\emph{Analisi dei Rischi} e lo \emph{Studio di Fattibilità} in diverse fasi.\\
Questo approccio è stato estremamente utile per far emergere rischi dovuti sia alla non piena maturità delle \emph{API}, sia ai limiti fisici del dispositivo in dotazione.


%**************************************************************
\section{Analisi preventiva dei rischi}

Durante la fase di analisi iniziale sono stati individuati alcuni possibili rischi a cui si potrà andare incontro.
Si è quindi proceduto a elaborare delle possibili soluzioni per far fronte a tali rischi.\\

\subsection{Rischi generali}
\begin{risk}{Immaturità di API/librerie/documentazione}
	\riskchance{Alta}
	\riskseverity{Media}
    \riskdescription{Le tecnologie adottate sono innovative e tuttora in fase di sviluppo, molte sono ancora segnalate come \emph{"Sperimentali e soggette a cambiamenti"}. Per questo le librerie usate potrebbero rivelarsi instabili o potrebbero mancare di adeguata documentazione}
    \risksolution{Iscrizione ai vari canali di segnalazione e supporto offerti da \emph{Google} per gli sviluppatori, sviluppo di piccoli esempi giocattolo per testare le funzionalità offerte dalle \emph{API} da cui è stata generata della documentazione interna}
    \label{risk:API-immaturity} 
\end{risk}
\begin{risk}{Limiti fisici del dispositivo}
	\riskchance{Alta}
	\riskseverity{Media}
    \riskdescription{Il dispositivo è dotato di sensori infrarossi e sfrutta la riflessione della luce per determinare la distanza dei punti che è in grado di individuare. Superfici riflettenti o molto scure possono compromettere la qualità della misurazione, allo stesso modo situazioni di illuminazione scarsa o assente}
    \risksolution{Accurata analisi della documentazione fornita dal produttore\footcite{site: https://developers.google.com/tango/overview/depth-perception}, test preventivi nelle situazioni critiche utilizzando una semplice applicazione di prova fornita da \emph{Google}\footcite{site: https://github.com/googlesamples/tango-examples-java/tree/master/java_point_cloud_example}}
    \label{risk:device-limit} 
\end{risk}

\subsection{Rischi specifici}
\begin{risk}{Difficoltà nel Motion Tracking}
	\riskchance{Media}
	\riskseverity{Alta}
    \riskdescription{Determinare la posizione e l'orientamento del dispositivo in maniera assoluta è fondamentale per permettere la ricostruzione dell'oggetto inquadrato. Il \emph{device} fornisce ad intervalli regolari la sua posizione tramite una tripletta di coordinate ed la sua rotazione rappresentata come un quaternione. La somma di piccoli errori relativi nella stima della posizione crea un fenomeno detto \emph{drifting} che comporta imporanti errori nella stima finale. Questo rischio può portare al fallimento del progetto, in quanto se non opportunamente mitigato renderebbe le ricostruzioni tridimensionali totalmente errate.}
    \risksolution{Si è per questo deciso di adottare una tecnica denominata \emph{Area Learning}. Il disposivo quindi riconoscerà alcune \emph{features}, ovvero dei punti fissi, rispetto ai quali determinerà la sua posizione.}
    \label{risk:motion-tracking} 
\end{risk}
\begin{risk}{Necessità di azioni specifiche da parte dell'utente}
	\riskchance{Alta}
	\riskseverity{Alta}
    \riskdescription{Tutte le applicazioni che usano la tecnologia \emph{Tango} iteragiscono strettamente con i movimenti e la posizione dell'utente. La scarsa diffusione di questa tecnologia fa si che la maggior parte dell'utenza non sia a conoscenza del comportamento che deve tenere. Azioni compiute dall'utente in maniera scorretta possono compromettere il buon funzionamento dell'applicazione.}
    \risksolution{Tutto lo sviluppo dell'applicazione deve tenere conto di questo fatto. Devono essere fornite chiare informazioni all'utente e si devono studiare soluzioni che non costringano l'\emph{user} ad un comportamento troppo antiintuitivo.}
    \label{risk:area-learning} 
\end{risk}
\begin{risk}{}
	\riskchance{}
	\riskseverity{}
    \riskdescription{}
    \risksolution{}
    \label{} 
\end{risk}


%**************************************************************
\section{Studio di fattibilità}
\subsection{Preventivo}
Prima di iniziare il progetto è stato effettuato un accurato studio di fattibilità basato sulla ricerca di progetti con funzionalità simili e sulla lettura delle numerose discussioni e pubblicazioni presenti nel \emph{web} a riguardo di \emph{Project Tango}.
xxxx













