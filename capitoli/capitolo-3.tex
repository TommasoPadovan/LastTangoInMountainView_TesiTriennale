% !TEX encoding = UTF-8
% !TEX TS-program = pdflatex
% !TEX root = ../tesi.tex
% !TEX spellcheck = it-IT

%**************************************************************
\chapter{Descrizione dello stage}
\label{cap:descrizione-stage}
%**************************************************************

\intro{xxxx Breve introduzione al capitolo}\\

%**************************************************************
\section{Introduzione al progetto}
Data la natura innovativa del progetto è stato necessario produrre diversi prototipi ed effettuare l'\emph{Analisi dei Rischi} e lo \emph{Studio di Fattibilità} in diverse fasi.\\
Questo approccio è stato estremamente utile per far emergere rischi dovuti sia alla non piena maturità delle \emph{API}, sia ai limiti fisici del dispositivo in dotazione.


%**************************************************************
\section{Analisi preventiva dei rischi}

Durante la fase di analisi iniziale sono stati individuati alcuni possibili rischi a cui si potrà andare incontro.
Si è quindi proceduto a elaborare delle possibili soluzioni per far fronte a tali rischi.\\

\subsection{Rischi generali}
\begin{risk}{Immaturità di API/librerie/documentazione}
	\riskchance{Alta}
	\riskseverity{Media}
    \riskdescription{Le tecnologie adottate sono innovative e tuttora in fase di sviluppo, molte sono ancora segnalate come \emph{"Sperimentali e soggette a cambiamenti"}. Per questo le librerie usate potrebbero rivelarsi instabili o potrebbero mancare di adeguata documentazione}
    \risksolution{Iscrizione ai vari canali di segnalazione e supporto offerti da \emph{Google} per gli sviluppatori, sviluppo di piccoli esempi giocattolo per testare le funzionalità offerte dalle \emph{API} da cui è stata generata della documentazione interna}
    \label{risk:API-immaturity} 
\end{risk}

\begin{risk}{Limiti fisici del dispositivo}
	\riskchance{Alta}
	\riskseverity{Media}
    \riskdescription{Il dispositivo è dotato di sensori infrarossi e sfrutta la riflessione della luce per determinare la distanza dei punti che è in grado di individuare. Superfici riflettenti o molto scure possono compromettere la qualità della misurazione, allo stesso modo situazioni di illuminazione scarsa o assente}
    \risksolution{Accurata analisi della documentazione fornita dal produttore\footcite{site: https://developers.google.com/tango/overview/depth-perception}, test preventivi nelle situazioni critiche utilizzando una semplice applicazione di prova fornita da \emph{Google}\footcite{site: https://github.com/googlesamples/tango-examples-java/tree/master/java_point_cloud_example}}
    \label{risk:device-limit} 
\end{risk}

\subsection{Rischi specifici}
\begin{risk}{Difficoltà nel Motion Tracking}
	\riskchance{Media}
	\riskseverity{Alta}
    \riskdescription{Determinare la posizione e l'orientamento del dispositivo in maniera assoluta è fondamentale per permettere la ricostruzione dell'oggetto inquadrato. Il \emph{device} fornisce ad intervalli regolari la sua posizione tramite una tripletta di coordinate ed la sua rotazione rappresentata come un quaternione. }
    \risksolution{}
    \label{} 
\end{risk}
\begin{risk}{}
	\riskchance{}
	\riskseverity{}
    \riskdescription{}
    \risksolution{}
    \label{} 
\end{risk}

%**************************************************************
\section{Requisiti e obiettivi}


%**************************************************************
\section{Pianificazione}