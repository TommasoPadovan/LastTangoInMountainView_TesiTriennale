% !TEX encoding = UTF-8
% !TEX TS-program = pdflatex
% !TEX root = ../tesi.tex
% !TEX spellcheck = it-IT

%**************************************************************
% Sommario
%**************************************************************
\cleardoublepage
\phantomsection
\pdfbookmark{Sommario}{Sommario}
\begingroup
\let\clearpage\relax
\let\cleardoublepage\relax
\let\cleardoublepage\relax

\chapter*{Sommario}

Il presente documento descrive il lavoro svolto durante il periodo di stage, della durata di circa trecentoventi ore, dal laureando \myName \ presso l'azienda Vic srl.\\
L'obiettivo principale era esplorare la possibilità di sfruttare gli innovativi dispositivi \emph{Tango} di \emph{Google} per produrre scansioni tridimensionali degli oggetti inquadrati.\\
In primo luogo era richiesta la progettazione e lo sviluppo di una applicazione \emph{mobile} \emph{Android} dotata di una minimale interfaccia grafica in grado di scannerizzare un oggetto, esportarlo in un formato portabile ed effettuare su quest'ultimo ottimizzazioni ed il calcolo del volume.\\
In secondo luogo era richiesto di massimizzare il riuso delle soluzioni \emph{OpenSource} presenti sul mercato. 
%\vfill
%
%\selectlanguage{english}
%\pdfbookmark{Abstract}{Abstract}
%\chapter*{Abstract}
%
%\selectlanguage{italian}

\endgroup			

\vfill

