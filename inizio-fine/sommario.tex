% !TEX encoding = UTF-8
% !TEX TS-program = pdflatex
% !TEX root = ../tesi.tex
% !TEX spellcheck = it-IT

%**************************************************************
% Sommario
%**************************************************************
\cleardoublepage
\phantomsection
\pdfbookmark{Sommario}{Sommario}
\begingroup
\let\clearpage\relax
\let\cleardoublepage\relax
\let\cleardoublepage\relax

\chapter*{Sommario}

Il presente documento descrive il lavoro svolto durante il periodo di stage, della durata di circa trecentoventi ore, dal laureando \myName \ presso l'azienda Vic srl.
Gli obbiettivi da raggiungere erano principalmente la progettazione e la codifica di un prototipo di applicazione in grado di sfruttare gli innovativi dispositivi \emph{Tango} di Google per produrre scansioni tridimensionali degli oggetti inquadrati.\\
In primo luogo era richiesto lo studio delle soluzioni \emph{OpenSource} già presenti nel mercato, al fine di massimizzare il riuso.
In secondo luogo era richiesta l'ideazione e la progettazione di una applicazione in grado di registrare ed eleborare i dati catturati dai sensori del \emph{tablet} utilizzando le \emph{API Tango} offerte da \emph{Google}. 
Il terzo obbiettivo era la progettazione e codifica di una applicazione dotata di una interfaccia grafica minimale, ma capace di ricostruire gli oggetti inquadrati ed inviare i dati, in formato \emph{Point Cloud} (\emph{.pcd}), ad un \emph{Server}.
In ultimo luogo era richiesto lo sviluppo dell'applicazione lato \emph{Server} allo scopo di effettuare ottimizzazioni sui \emph{Point Cloud} ed il calcolo del volume. Tali operazioni sono state realizzate lato \emph{Server} in quanto sarebbero troppo onerose per un dispositivo \emph{mobile}.

%\vfill
%
%\selectlanguage{english}
%\pdfbookmark{Abstract}{Abstract}
%\chapter*{Abstract}
%
%\selectlanguage{italian}

\endgroup			

\vfill

